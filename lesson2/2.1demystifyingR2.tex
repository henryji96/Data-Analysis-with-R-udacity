\documentclass[]{article}
\usepackage{lmodern}
\usepackage{amssymb,amsmath}
\usepackage{ifxetex,ifluatex}
\usepackage{fixltx2e} % provides \textsubscript
\ifnum 0\ifxetex 1\fi\ifluatex 1\fi=0 % if pdftex
  \usepackage[T1]{fontenc}
  \usepackage[utf8]{inputenc}
\else % if luatex or xelatex
  \ifxetex
    \usepackage{mathspec}
  \else
    \usepackage{fontspec}
  \fi
  \defaultfontfeatures{Ligatures=TeX,Scale=MatchLowercase}
\fi
% use upquote if available, for straight quotes in verbatim environments
\IfFileExists{upquote.sty}{\usepackage{upquote}}{}
% use microtype if available
\IfFileExists{microtype.sty}{%
\usepackage{microtype}
\UseMicrotypeSet[protrusion]{basicmath} % disable protrusion for tt fonts
}{}
\usepackage[margin=1in]{geometry}
\usepackage{hyperref}
\hypersetup{unicode=true,
            pdfborder={0 0 0},
            breaklinks=true}
\urlstyle{same}  % don't use monospace font for urls
\usepackage{color}
\usepackage{fancyvrb}
\newcommand{\VerbBar}{|}
\newcommand{\VERB}{\Verb[commandchars=\\\{\}]}
\DefineVerbatimEnvironment{Highlighting}{Verbatim}{commandchars=\\\{\}}
% Add ',fontsize=\small' for more characters per line
\usepackage{framed}
\definecolor{shadecolor}{RGB}{248,248,248}
\newenvironment{Shaded}{\begin{snugshade}}{\end{snugshade}}
\newcommand{\KeywordTok}[1]{\textcolor[rgb]{0.13,0.29,0.53}{\textbf{#1}}}
\newcommand{\DataTypeTok}[1]{\textcolor[rgb]{0.13,0.29,0.53}{#1}}
\newcommand{\DecValTok}[1]{\textcolor[rgb]{0.00,0.00,0.81}{#1}}
\newcommand{\BaseNTok}[1]{\textcolor[rgb]{0.00,0.00,0.81}{#1}}
\newcommand{\FloatTok}[1]{\textcolor[rgb]{0.00,0.00,0.81}{#1}}
\newcommand{\ConstantTok}[1]{\textcolor[rgb]{0.00,0.00,0.00}{#1}}
\newcommand{\CharTok}[1]{\textcolor[rgb]{0.31,0.60,0.02}{#1}}
\newcommand{\SpecialCharTok}[1]{\textcolor[rgb]{0.00,0.00,0.00}{#1}}
\newcommand{\StringTok}[1]{\textcolor[rgb]{0.31,0.60,0.02}{#1}}
\newcommand{\VerbatimStringTok}[1]{\textcolor[rgb]{0.31,0.60,0.02}{#1}}
\newcommand{\SpecialStringTok}[1]{\textcolor[rgb]{0.31,0.60,0.02}{#1}}
\newcommand{\ImportTok}[1]{#1}
\newcommand{\CommentTok}[1]{\textcolor[rgb]{0.56,0.35,0.01}{\textit{#1}}}
\newcommand{\DocumentationTok}[1]{\textcolor[rgb]{0.56,0.35,0.01}{\textbf{\textit{#1}}}}
\newcommand{\AnnotationTok}[1]{\textcolor[rgb]{0.56,0.35,0.01}{\textbf{\textit{#1}}}}
\newcommand{\CommentVarTok}[1]{\textcolor[rgb]{0.56,0.35,0.01}{\textbf{\textit{#1}}}}
\newcommand{\OtherTok}[1]{\textcolor[rgb]{0.56,0.35,0.01}{#1}}
\newcommand{\FunctionTok}[1]{\textcolor[rgb]{0.00,0.00,0.00}{#1}}
\newcommand{\VariableTok}[1]{\textcolor[rgb]{0.00,0.00,0.00}{#1}}
\newcommand{\ControlFlowTok}[1]{\textcolor[rgb]{0.13,0.29,0.53}{\textbf{#1}}}
\newcommand{\OperatorTok}[1]{\textcolor[rgb]{0.81,0.36,0.00}{\textbf{#1}}}
\newcommand{\BuiltInTok}[1]{#1}
\newcommand{\ExtensionTok}[1]{#1}
\newcommand{\PreprocessorTok}[1]{\textcolor[rgb]{0.56,0.35,0.01}{\textit{#1}}}
\newcommand{\AttributeTok}[1]{\textcolor[rgb]{0.77,0.63,0.00}{#1}}
\newcommand{\RegionMarkerTok}[1]{#1}
\newcommand{\InformationTok}[1]{\textcolor[rgb]{0.56,0.35,0.01}{\textbf{\textit{#1}}}}
\newcommand{\WarningTok}[1]{\textcolor[rgb]{0.56,0.35,0.01}{\textbf{\textit{#1}}}}
\newcommand{\AlertTok}[1]{\textcolor[rgb]{0.94,0.16,0.16}{#1}}
\newcommand{\ErrorTok}[1]{\textcolor[rgb]{0.64,0.00,0.00}{\textbf{#1}}}
\newcommand{\NormalTok}[1]{#1}
\usepackage{graphicx,grffile}
\makeatletter
\def\maxwidth{\ifdim\Gin@nat@width>\linewidth\linewidth\else\Gin@nat@width\fi}
\def\maxheight{\ifdim\Gin@nat@height>\textheight\textheight\else\Gin@nat@height\fi}
\makeatother
% Scale images if necessary, so that they will not overflow the page
% margins by default, and it is still possible to overwrite the defaults
% using explicit options in \includegraphics[width, height, ...]{}
\setkeys{Gin}{width=\maxwidth,height=\maxheight,keepaspectratio}
\IfFileExists{parskip.sty}{%
\usepackage{parskip}
}{% else
\setlength{\parindent}{0pt}
\setlength{\parskip}{6pt plus 2pt minus 1pt}
}
\setlength{\emergencystretch}{3em}  % prevent overfull lines
\providecommand{\tightlist}{%
  \setlength{\itemsep}{0pt}\setlength{\parskip}{0pt}}
\setcounter{secnumdepth}{0}
% Redefines (sub)paragraphs to behave more like sections
\ifx\paragraph\undefined\else
\let\oldparagraph\paragraph
\renewcommand{\paragraph}[1]{\oldparagraph{#1}\mbox{}}
\fi
\ifx\subparagraph\undefined\else
\let\oldsubparagraph\subparagraph
\renewcommand{\subparagraph}[1]{\oldsubparagraph{#1}\mbox{}}
\fi

%%% Use protect on footnotes to avoid problems with footnotes in titles
\let\rmarkdownfootnote\footnote%
\def\footnote{\protect\rmarkdownfootnote}

%%% Change title format to be more compact
\usepackage{titling}

% Create subtitle command for use in maketitle
\newcommand{\subtitle}[1]{
  \posttitle{
    \begin{center}\large#1\end{center}
    }
}

\setlength{\droptitle}{-2em}

  \title{}
    \pretitle{\vspace{\droptitle}}
  \posttitle{}
    \author{}
    \preauthor{}\postauthor{}
    \date{}
    \predate{}\postdate{}
  

\begin{document}

\section{Demystifying R Part 2}\label{demystifying-r-part-2}

You might see a warning message just above this file. Something
like\ldots{} ``R Markdown requires the knitr package (version 1.2 or
higher)'' Don't worry about this for now. We'll address it at the end of
this file.

\begin{enumerate}
\def\labelenumi{\arabic{enumi}.}
\tightlist
\item
  Run the following command to see what it does.
\end{enumerate}

\begin{Shaded}
\begin{Highlighting}[]
\KeywordTok{summary}\NormalTok{(mtcars)}
\end{Highlighting}
\end{Shaded}

\begin{verbatim}
##       mpg             cyl             disp             hp       
##  Min.   :10.40   Min.   :4.000   Min.   : 71.1   Min.   : 52.0  
##  1st Qu.:15.43   1st Qu.:4.000   1st Qu.:120.8   1st Qu.: 96.5  
##  Median :19.20   Median :6.000   Median :196.3   Median :123.0  
##  Mean   :20.09   Mean   :6.188   Mean   :230.7   Mean   :146.7  
##  3rd Qu.:22.80   3rd Qu.:8.000   3rd Qu.:326.0   3rd Qu.:180.0  
##  Max.   :33.90   Max.   :8.000   Max.   :472.0   Max.   :335.0  
##       drat             wt             qsec             vs        
##  Min.   :2.760   Min.   :1.513   Min.   :14.50   Min.   :0.0000  
##  1st Qu.:3.080   1st Qu.:2.581   1st Qu.:16.89   1st Qu.:0.0000  
##  Median :3.695   Median :3.325   Median :17.71   Median :0.0000  
##  Mean   :3.597   Mean   :3.217   Mean   :17.85   Mean   :0.4375  
##  3rd Qu.:3.920   3rd Qu.:3.610   3rd Qu.:18.90   3rd Qu.:1.0000  
##  Max.   :4.930   Max.   :5.424   Max.   :22.90   Max.   :1.0000  
##        am              gear            carb      
##  Min.   :0.0000   Min.   :3.000   Min.   :1.000  
##  1st Qu.:0.0000   1st Qu.:3.000   1st Qu.:2.000  
##  Median :0.0000   Median :4.000   Median :2.000  
##  Mean   :0.4062   Mean   :3.688   Mean   :2.812  
##  3rd Qu.:1.0000   3rd Qu.:4.000   3rd Qu.:4.000  
##  Max.   :1.0000   Max.   :5.000   Max.   :8.000
\end{verbatim}

\begin{Shaded}
\begin{Highlighting}[]
\KeywordTok{str}\NormalTok{(mtcars)}
\end{Highlighting}
\end{Shaded}

\begin{verbatim}
## 'data.frame':    32 obs. of  11 variables:
##  $ mpg : num  21 21 22.8 21.4 18.7 18.1 14.3 24.4 22.8 19.2 ...
##  $ cyl : num  6 6 4 6 8 6 8 4 4 6 ...
##  $ disp: num  160 160 108 258 360 ...
##  $ hp  : num  110 110 93 110 175 105 245 62 95 123 ...
##  $ drat: num  3.9 3.9 3.85 3.08 3.15 2.76 3.21 3.69 3.92 3.92 ...
##  $ wt  : num  2.62 2.88 2.32 3.21 3.44 ...
##  $ qsec: num  16.5 17 18.6 19.4 17 ...
##  $ vs  : num  0 0 1 1 0 1 0 1 1 1 ...
##  $ am  : num  1 1 1 0 0 0 0 0 0 0 ...
##  $ gear: num  4 4 4 3 3 3 3 4 4 4 ...
##  $ carb: num  4 4 1 1 2 1 4 2 2 4 ...
\end{verbatim}

\begin{Shaded}
\begin{Highlighting}[]
\KeywordTok{dim}\NormalTok{(mtcars)}
\end{Highlighting}
\end{Shaded}

\begin{verbatim}
## [1] 32 11
\end{verbatim}

If you know about quantiles, then the output should look familiar. If
not, you probably recognize the min (minimum), median, mean, and max
(maximum). We'll go over quantiles in Lesson 3 so don't worry if the
output seems overwhelming.

The str() and summary() functions are helpful commands when working with
a new data set. The str() function gives us the variable names and their
types. The summary() function gives us an idea of the values a variable
can take on.

\begin{enumerate}
\def\labelenumi{\arabic{enumi}.}
\setcounter{enumi}{1}
\tightlist
\item
  In 2013, the average mpg (miles per gallon) for a car was 23 mpg. The
  car models in the mtcars data set come from the year 1973-1974. Subset
  the data so that you create a new data frame that contains cars that
  get 23 or more mpg (miles per gallon). Save it to a new data frame
  called efficient.
\end{enumerate}

\begin{Shaded}
\begin{Highlighting}[]
\NormalTok{efficient<-}\KeywordTok{subset}\NormalTok{(mtcars,mpg}\OperatorTok{>=}\DecValTok{23} \OperatorTok{&}\StringTok{ }\NormalTok{vs}\OperatorTok{==}\DecValTok{1}\NormalTok{)}
\NormalTok{efficient}
\end{Highlighting}
\end{Shaded}

\begin{verbatim}
##                 mpg cyl  disp  hp drat    wt  qsec vs am gear carb
## Merc 240D      24.4   4 146.7  62 3.69 3.190 20.00  1  0    4    2
## Fiat 128       32.4   4  78.7  66 4.08 2.200 19.47  1  1    4    1
## Honda Civic    30.4   4  75.7  52 4.93 1.615 18.52  1  1    4    2
## Toyota Corolla 33.9   4  71.1  65 4.22 1.835 19.90  1  1    4    1
## Fiat X1-9      27.3   4  79.0  66 4.08 1.935 18.90  1  1    4    1
## Lotus Europa   30.4   4  95.1 113 3.77 1.513 16.90  1  1    5    2
\end{verbatim}

\begin{enumerate}
\def\labelenumi{\arabic{enumi}.}
\setcounter{enumi}{2}
\tightlist
\item
  How many cars get more than 23 mpg? Use one of the commands you
  learned in the demystifying.R to answer this question.
\end{enumerate}

\begin{Shaded}
\begin{Highlighting}[]
\KeywordTok{dim}\NormalTok{(efficient)}
\end{Highlighting}
\end{Shaded}

\begin{verbatim}
## [1]  6 11
\end{verbatim}

\begin{Shaded}
\begin{Highlighting}[]
\NormalTok{lt23<-mtcars[mtcars}\OperatorTok{$}\NormalTok{mpg}\OperatorTok{>=}\DecValTok{23}\NormalTok{,]}
\KeywordTok{dim}\NormalTok{(lt23)}
\end{Highlighting}
\end{Shaded}

\begin{verbatim}
## [1]  7 11
\end{verbatim}

\begin{enumerate}
\def\labelenumi{\arabic{enumi}.}
\setcounter{enumi}{3}
\tightlist
\item
  We can also use logical operators to find out which car(s) get greater
  than 30 miles per gallon (mpg) and have more than 100 raw horsepower.
\end{enumerate}

\begin{Shaded}
\begin{Highlighting}[]
\KeywordTok{subset}\NormalTok{(mtcars, mpg }\OperatorTok{>}\StringTok{ }\DecValTok{30} \OperatorTok{&}\StringTok{ }\NormalTok{hp }\OperatorTok{>}\StringTok{ }\DecValTok{100}\NormalTok{)}
\end{Highlighting}
\end{Shaded}

\begin{verbatim}
##               mpg cyl disp  hp drat    wt qsec vs am gear carb
## Lotus Europa 30.4   4 95.1 113 3.77 1.513 16.9  1  1    5    2
\end{verbatim}

\begin{Shaded}
\begin{Highlighting}[]
\NormalTok{mtcars[mtcars}\OperatorTok{$}\NormalTok{mpg}\OperatorTok{>}\DecValTok{30}\OperatorTok{&}\NormalTok{mtcars}\OperatorTok{$}\NormalTok{hp}\OperatorTok{>}\DecValTok{100}\NormalTok{,]}
\end{Highlighting}
\end{Shaded}

\begin{verbatim}
##               mpg cyl disp  hp drat    wt qsec vs am gear carb
## Lotus Europa 30.4   4 95.1 113 3.77 1.513 16.9  1  1    5    2
\end{verbatim}

There's only one car that gets more than 30 mpg and 100 hp.

\begin{enumerate}
\def\labelenumi{\arabic{enumi}.}
\setcounter{enumi}{4}
\tightlist
\item
  What do you think this code does? Scroll down for the answer.
\end{enumerate}

\begin{Shaded}
\begin{Highlighting}[]
\KeywordTok{subset}\NormalTok{(mtcars, mpg }\OperatorTok{<}\StringTok{ }\DecValTok{14} \OperatorTok{|}\StringTok{ }\NormalTok{disp }\OperatorTok{>}\StringTok{ }\DecValTok{390}\NormalTok{)}
\end{Highlighting}
\end{Shaded}

\begin{verbatim}
##                      mpg cyl disp  hp drat    wt  qsec vs am gear carb
## Cadillac Fleetwood  10.4   8  472 205 2.93 5.250 17.98  0  0    3    4
## Lincoln Continental 10.4   8  460 215 3.00 5.424 17.82  0  0    3    4
## Chrysler Imperial   14.7   8  440 230 3.23 5.345 17.42  0  0    3    4
## Camaro Z28          13.3   8  350 245 3.73 3.840 15.41  0  0    3    4
## Pontiac Firebird    19.2   8  400 175 3.08 3.845 17.05  0  0    3    2
\end{verbatim}

Note: You may be familiar with the \textbar{}\textbar{} operator in
Java. R uses one single \& for the logical operator AND. It also uses
one \textbar{} for the logical operator OR.

The command above creates a data frame of cars that have mpg less than
14 OR a displacement of more than 390. Only one of the conditions for a
car needs to be satisfied so that the car makes it into the subset. Any
of the cars that fit the criteria are printed to the console.

Now you try some.

\begin{enumerate}
\def\labelenumi{\arabic{enumi}.}
\setcounter{enumi}{5}
\tightlist
\item
  Print the cars that have a 1/4 mile time (qsec) less than or equal to
  16.90 seconds to the console.
\end{enumerate}

\begin{Shaded}
\begin{Highlighting}[]
\KeywordTok{subset}\NormalTok{(mtcars,qsec}\OperatorTok{<=}\FloatTok{16.9}\NormalTok{)}
\end{Highlighting}
\end{Shaded}

\begin{verbatim}
##                   mpg cyl  disp  hp drat    wt  qsec vs am gear carb
## Mazda RX4        21.0   6 160.0 110 3.90 2.620 16.46  0  1    4    4
## Duster 360       14.3   8 360.0 245 3.21 3.570 15.84  0  0    3    4
## Dodge Challenger 15.5   8 318.0 150 2.76 3.520 16.87  0  0    3    2
## Camaro Z28       13.3   8 350.0 245 3.73 3.840 15.41  0  0    3    4
## Porsche 914-2    26.0   4 120.3  91 4.43 2.140 16.70  0  1    5    2
## Lotus Europa     30.4   4  95.1 113 3.77 1.513 16.90  1  1    5    2
## Ford Pantera L   15.8   8 351.0 264 4.22 3.170 14.50  0  1    5    4
## Ferrari Dino     19.7   6 145.0 175 3.62 2.770 15.50  0  1    5    6
## Maserati Bora    15.0   8 301.0 335 3.54 3.570 14.60  0  1    5    8
\end{verbatim}

\begin{enumerate}
\def\labelenumi{\arabic{enumi}.}
\setcounter{enumi}{6}
\tightlist
\item
  Save the subset of cars that weigh under 2000 pounds (weight is
  measured in lb/1000) to a variable called lightCars. Print the numbers
  of cars and the subset to the console.
\end{enumerate}

\begin{Shaded}
\begin{Highlighting}[]
\NormalTok{lightCars<-}\KeywordTok{subset}\NormalTok{(mtcars,(mtcars}\OperatorTok{$}\NormalTok{wt}\OperatorTok{*}\DecValTok{1000}\NormalTok{)}\OperatorTok{<=}\DecValTok{2000}\NormalTok{)}
\KeywordTok{dim}\NormalTok{(lightCars)}
\end{Highlighting}
\end{Shaded}

\begin{verbatim}
## [1]  4 11
\end{verbatim}

\begin{enumerate}
\def\labelenumi{\arabic{enumi}.}
\setcounter{enumi}{7}
\tightlist
\item
  You can also create new variables in a data frame. Let's say you
  wanted to have the year of each car's model. We can create the
  variable mtcars\$year. Here we'll assume that all of the models were
  from 1974. Run the code below.
\end{enumerate}

\begin{Shaded}
\begin{Highlighting}[]
\NormalTok{mtcars}\OperatorTok{$}\NormalTok{year <-}\StringTok{ }\KeywordTok{c}\NormalTok{(}\DecValTok{1974}\NormalTok{,}\DecValTok{2001}\NormalTok{)}
\KeywordTok{dim}\NormalTok{(mtcars)}
\end{Highlighting}
\end{Shaded}

\begin{verbatim}
## [1] 32 12
\end{verbatim}

\begin{Shaded}
\begin{Highlighting}[]
\KeywordTok{head}\NormalTok{(mtcars,}\DecValTok{4}\NormalTok{)}
\end{Highlighting}
\end{Shaded}

\begin{verbatim}
##                 mpg cyl disp  hp drat    wt  qsec vs am gear carb year
## Mazda RX4      21.0   6  160 110 3.90 2.620 16.46  0  1    4    4 1974
## Mazda RX4 Wag  21.0   6  160 110 3.90 2.875 17.02  0  1    4    4 2001
## Datsun 710     22.8   4  108  93 3.85 2.320 18.61  1  1    4    1 1974
## Hornet 4 Drive 21.4   6  258 110 3.08 3.215 19.44  1  0    3    1 2001
\end{verbatim}

Notice how the number of variables changed in the work space. You can
also see the result by double clicking on mtcars in the workspace and
examining the data in a table.

To drop a variable, subset the data frame and select the variable you
want to drop with a negative sign in front of it.

\begin{Shaded}
\begin{Highlighting}[]
\NormalTok{mtcars <-}\StringTok{ }\KeywordTok{subset}\NormalTok{(mtcars, }\DataTypeTok{select =} \OperatorTok{-}\NormalTok{year)}
\KeywordTok{dim}\NormalTok{(mtcars)}
\end{Highlighting}
\end{Shaded}

\begin{verbatim}
## [1] 32 11
\end{verbatim}

\begin{Shaded}
\begin{Highlighting}[]
\KeywordTok{head}\NormalTok{(mtcars,}\DecValTok{2}\NormalTok{)}
\end{Highlighting}
\end{Shaded}

\begin{verbatim}
##               mpg cyl disp  hp drat    wt  qsec vs am gear carb
## Mazda RX4      21   6  160 110  3.9 2.620 16.46  0  1    4    4
## Mazda RX4 Wag  21   6  160 110  3.9 2.875 17.02  0  1    4    4
\end{verbatim}

Notice, we are back to 11 variables in the data frame.

\begin{enumerate}
\def\labelenumi{\arabic{enumi}.}
\setcounter{enumi}{8}
\tightlist
\item
  What do you think this code does? Run it to find out.
\end{enumerate}

\begin{Shaded}
\begin{Highlighting}[]
\NormalTok{mtcars}\OperatorTok{$}\NormalTok{year <-}\StringTok{ }\KeywordTok{c}\NormalTok{(}\DecValTok{1973}\NormalTok{, }\DecValTok{1974}\NormalTok{)}
\NormalTok{mtcars}
\end{Highlighting}
\end{Shaded}

\begin{verbatim}
##                      mpg cyl  disp  hp drat    wt  qsec vs am gear carb
## Mazda RX4           21.0   6 160.0 110 3.90 2.620 16.46  0  1    4    4
## Mazda RX4 Wag       21.0   6 160.0 110 3.90 2.875 17.02  0  1    4    4
## Datsun 710          22.8   4 108.0  93 3.85 2.320 18.61  1  1    4    1
## Hornet 4 Drive      21.4   6 258.0 110 3.08 3.215 19.44  1  0    3    1
## Hornet Sportabout   18.7   8 360.0 175 3.15 3.440 17.02  0  0    3    2
## Valiant             18.1   6 225.0 105 2.76 3.460 20.22  1  0    3    1
## Duster 360          14.3   8 360.0 245 3.21 3.570 15.84  0  0    3    4
## Merc 240D           24.4   4 146.7  62 3.69 3.190 20.00  1  0    4    2
## Merc 230            22.8   4 140.8  95 3.92 3.150 22.90  1  0    4    2
## Merc 280            19.2   6 167.6 123 3.92 3.440 18.30  1  0    4    4
## Merc 280C           17.8   6 167.6 123 3.92 3.440 18.90  1  0    4    4
## Merc 450SE          16.4   8 275.8 180 3.07 4.070 17.40  0  0    3    3
## Merc 450SL          17.3   8 275.8 180 3.07 3.730 17.60  0  0    3    3
## Merc 450SLC         15.2   8 275.8 180 3.07 3.780 18.00  0  0    3    3
## Cadillac Fleetwood  10.4   8 472.0 205 2.93 5.250 17.98  0  0    3    4
## Lincoln Continental 10.4   8 460.0 215 3.00 5.424 17.82  0  0    3    4
## Chrysler Imperial   14.7   8 440.0 230 3.23 5.345 17.42  0  0    3    4
## Fiat 128            32.4   4  78.7  66 4.08 2.200 19.47  1  1    4    1
## Honda Civic         30.4   4  75.7  52 4.93 1.615 18.52  1  1    4    2
## Toyota Corolla      33.9   4  71.1  65 4.22 1.835 19.90  1  1    4    1
## Toyota Corona       21.5   4 120.1  97 3.70 2.465 20.01  1  0    3    1
## Dodge Challenger    15.5   8 318.0 150 2.76 3.520 16.87  0  0    3    2
## AMC Javelin         15.2   8 304.0 150 3.15 3.435 17.30  0  0    3    2
## Camaro Z28          13.3   8 350.0 245 3.73 3.840 15.41  0  0    3    4
## Pontiac Firebird    19.2   8 400.0 175 3.08 3.845 17.05  0  0    3    2
## Fiat X1-9           27.3   4  79.0  66 4.08 1.935 18.90  1  1    4    1
## Porsche 914-2       26.0   4 120.3  91 4.43 2.140 16.70  0  1    5    2
## Lotus Europa        30.4   4  95.1 113 3.77 1.513 16.90  1  1    5    2
## Ford Pantera L      15.8   8 351.0 264 4.22 3.170 14.50  0  1    5    4
## Ferrari Dino        19.7   6 145.0 175 3.62 2.770 15.50  0  1    5    6
## Maserati Bora       15.0   8 301.0 335 3.54 3.570 14.60  0  1    5    8
## Volvo 142E          21.4   4 121.0 109 4.11 2.780 18.60  1  1    4    2
##                     year
## Mazda RX4           1973
## Mazda RX4 Wag       1974
## Datsun 710          1973
## Hornet 4 Drive      1974
## Hornet Sportabout   1973
## Valiant             1974
## Duster 360          1973
## Merc 240D           1974
## Merc 230            1973
## Merc 280            1974
## Merc 280C           1973
## Merc 450SE          1974
## Merc 450SL          1973
## Merc 450SLC         1974
## Cadillac Fleetwood  1973
## Lincoln Continental 1974
## Chrysler Imperial   1973
## Fiat 128            1974
## Honda Civic         1973
## Toyota Corolla      1974
## Toyota Corona       1973
## Dodge Challenger    1974
## AMC Javelin         1973
## Camaro Z28          1974
## Pontiac Firebird    1973
## Fiat X1-9           1974
## Porsche 914-2       1973
## Lotus Europa        1974
## Ford Pantera L      1973
## Ferrari Dino        1974
## Maserati Bora       1973
## Volvo 142E          1974
\end{verbatim}

Open the table of values to see what values year takes on.

Drop the year variable from the data set.

\begin{Shaded}
\begin{Highlighting}[]
\NormalTok{mtcars<-}\KeywordTok{subset}\NormalTok{(mtcars,}\DataTypeTok{select=}\OperatorTok{-}\NormalTok{year)}
\KeywordTok{dim}\NormalTok{(mtcars)}
\end{Highlighting}
\end{Shaded}

\begin{verbatim}
## [1] 32 11
\end{verbatim}

\begin{enumerate}
\def\labelenumi{\arabic{enumi}.}
\setcounter{enumi}{9}
\tightlist
\item
  Now you are going to get a preview of ifelse(). For those new to
  programming this example may be confusing. See if you can understand
  the code by running the commands one line at a time. Read the output
  and make sense of what the code is doing at each step.
\end{enumerate}

If you are having trouble don't worry, we will review the ifelse
statement at the end of Lesson 3. You won't be quizzed on it, and it's
not essential to keep going in this course. We just want you to try to
get familiar with more code.

\begin{Shaded}
\begin{Highlighting}[]
\NormalTok{mtcars}\OperatorTok{$}\NormalTok{wt}
\end{Highlighting}
\end{Shaded}

\begin{verbatim}
##  [1] 2.620 2.875 2.320 3.215 3.440 3.460 3.570 3.190 3.150 3.440 3.440
## [12] 4.070 3.730 3.780 5.250 5.424 5.345 2.200 1.615 1.835 2.465 3.520
## [23] 3.435 3.840 3.845 1.935 2.140 1.513 3.170 2.770 3.570 2.780
\end{verbatim}

\begin{Shaded}
\begin{Highlighting}[]
\NormalTok{cond <-}\StringTok{ }\NormalTok{mtcars}\OperatorTok{$}\NormalTok{wt }\OperatorTok{<}\StringTok{ }\DecValTok{3}
\NormalTok{cond}
\end{Highlighting}
\end{Shaded}

\begin{verbatim}
##  [1]  TRUE  TRUE  TRUE FALSE FALSE FALSE FALSE FALSE FALSE FALSE FALSE
## [12] FALSE FALSE FALSE FALSE FALSE FALSE  TRUE  TRUE  TRUE  TRUE FALSE
## [23] FALSE FALSE FALSE  TRUE  TRUE  TRUE FALSE  TRUE FALSE  TRUE
\end{verbatim}

\begin{Shaded}
\begin{Highlighting}[]
\NormalTok{mtcars}\OperatorTok{$}\NormalTok{weight_class <-}\StringTok{ }\KeywordTok{ifelse}\NormalTok{(cond, }\StringTok{'light'}\NormalTok{, }\StringTok{'average'}\NormalTok{)}
\NormalTok{mtcars}\OperatorTok{$}\NormalTok{weight_class}
\end{Highlighting}
\end{Shaded}

\begin{verbatim}
##  [1] "light"   "light"   "light"   "average" "average" "average" "average"
##  [8] "average" "average" "average" "average" "average" "average" "average"
## [15] "average" "average" "average" "light"   "light"   "light"   "light"  
## [22] "average" "average" "average" "average" "light"   "light"   "light"  
## [29] "average" "light"   "average" "light"
\end{verbatim}

\begin{Shaded}
\begin{Highlighting}[]
\NormalTok{cond <-}\StringTok{ }\NormalTok{mtcars}\OperatorTok{$}\NormalTok{wt }\OperatorTok{>}\StringTok{ }\FloatTok{3.5}
\NormalTok{mtcars}\OperatorTok{$}\NormalTok{weight_class <-}\StringTok{ }\KeywordTok{ifelse}\NormalTok{(cond, }\StringTok{'heavy'}\NormalTok{, mtcars}\OperatorTok{$}\NormalTok{weight_class)}
\NormalTok{mtcars}\OperatorTok{$}\NormalTok{weight_class}
\end{Highlighting}
\end{Shaded}

\begin{verbatim}
##  [1] "light"   "light"   "light"   "average" "average" "average" "heavy"  
##  [8] "average" "average" "average" "average" "heavy"   "heavy"   "heavy"  
## [15] "heavy"   "heavy"   "heavy"   "light"   "light"   "light"   "light"  
## [22] "heavy"   "average" "heavy"   "heavy"   "light"   "light"   "light"  
## [29] "average" "light"   "heavy"   "light"
\end{verbatim}

You have some variables in your workspace or environment like `cond' and
efficient. You want to be careful that you don't bring in too much data
into R at once since R will hold all the data in working memory. We have
nothing to worry about here, but let's delete those variables from the
work space.

\begin{Shaded}
\begin{Highlighting}[]
\KeywordTok{rm}\NormalTok{(lt23)}
\end{Highlighting}
\end{Shaded}

Save this file if you haven't done so yet.

You'll have the opportunity to create one Rmd file for the final project
in this class and submit the Rmd file and knitted output (or HTML file).
You'll need the knitr package to do that so let's install that now.
\textbf{Uncomment} the following two lines of code and run them.

\begin{Shaded}
\begin{Highlighting}[]
 \KeywordTok{install.packages}\NormalTok{(}\StringTok{'knitr'}\NormalTok{, }\DataTypeTok{dependencies =}\NormalTok{ T,}\DataTypeTok{repos=}\StringTok{"https://cran.rstudio.com"}\NormalTok{)}
\end{Highlighting}
\end{Shaded}

\begin{verbatim}
## 
## The downloaded binary packages are in
##  /var/folders/1b/bcp74kdj7f9c3l79bh_nclmm0000gn/T//RtmpZBNpTF/downloaded_packages
\end{verbatim}

\begin{Shaded}
\begin{Highlighting}[]
 \KeywordTok{library}\NormalTok{(knitr)}
\end{Highlighting}
\end{Shaded}

Once you've installed knitr, \textbf{comment} out the two lines of code
above. When you click the \textbf{Knit HTML} button a web page will be
generated that includes both content (text and text formatting from
Markdown) as well as the output of any embedded R code chunks within the
document.

You've reached the end of the file so now it's time to write some code
to answer a question to continue on in Lesson 2.

Which car(s) have an mpg (miles per gallon) greater than or equal to 30
OR hp (horsepower) less than 60? Create an R chunk of code to answer the
question.

\begin{Shaded}
\begin{Highlighting}[]
\NormalTok{answer<-}\KeywordTok{subset}\NormalTok{(mtcars,mpg}\OperatorTok{>=}\DecValTok{30}\OperatorTok{|}\NormalTok{hp}\OperatorTok{<}\DecValTok{60}\NormalTok{)}
\KeywordTok{dim}\NormalTok{(answer)}
\end{Highlighting}
\end{Shaded}

\begin{verbatim}
## [1]  4 12
\end{verbatim}

\begin{Shaded}
\begin{Highlighting}[]
\NormalTok{answer}
\end{Highlighting}
\end{Shaded}

\begin{verbatim}
##                 mpg cyl disp  hp drat    wt  qsec vs am gear carb
## Fiat 128       32.4   4 78.7  66 4.08 2.200 19.47  1  1    4    1
## Honda Civic    30.4   4 75.7  52 4.93 1.615 18.52  1  1    4    2
## Toyota Corolla 33.9   4 71.1  65 4.22 1.835 19.90  1  1    4    1
## Lotus Europa   30.4   4 95.1 113 3.77 1.513 16.90  1  1    5    2
##                weight_class
## Fiat 128              light
## Honda Civic           light
## Toyota Corolla        light
## Lotus Europa          light
\end{verbatim}

\begin{Shaded}
\begin{Highlighting}[]
\NormalTok{mtcars[mtcars}\OperatorTok{$}\NormalTok{mpg}\OperatorTok{>=}\DecValTok{30}\OperatorTok{|}\NormalTok{mtcars}\OperatorTok{$}\NormalTok{hp}\OperatorTok{<}\DecValTok{60}\NormalTok{,]}
\end{Highlighting}
\end{Shaded}

\begin{verbatim}
##                 mpg cyl disp  hp drat    wt  qsec vs am gear carb
## Fiat 128       32.4   4 78.7  66 4.08 2.200 19.47  1  1    4    1
## Honda Civic    30.4   4 75.7  52 4.93 1.615 18.52  1  1    4    2
## Toyota Corolla 33.9   4 71.1  65 4.22 1.835 19.90  1  1    4    1
## Lotus Europa   30.4   4 95.1 113 3.77 1.513 16.90  1  1    5    2
##                weight_class
## Fiat 128              light
## Honda Civic           light
## Toyota Corolla        light
## Lotus Europa          light
\end{verbatim}

Once you have the answer, go the
\href{https://www.udacity.com/course/viewer\#!/c-ud651/l-729069797/e-804129319/m-811719066}{Udacity
website} to continue with Lesson 2.

Note: You use brackets around text followed by two parentheses to create
a link. There must be no spaces between the brackets and the
parentheses. Paste or type the link into the parentheses. This also
works on the discussions!

And if you want to see all of your HARD WORK from this file, click the
\textbf{KNIT HTML} button now. (You may or may not need to restart R).

\section{CONGRATULATIONS}\label{congratulations}

\paragraph{You'll be exploring data soon with your new knowledge of
R.}\label{youll-be-exploring-data-soon-with-your-new-knowledge-of-r.}


\end{document}
